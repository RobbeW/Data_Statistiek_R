\documentclass[dvipsnames,tikz]{standalone}
\usepackage{amsmath}
\usepackage{arevmath}
\usepackage{xcolor}
\usepackage{tikz}
\usetikzlibrary{calc}
\usetikzlibrary{decorations.pathreplacing,calligraphy,3d}
\usetikzlibrary{lindenmayersystems}

\tikzset{main/.style={draw=black, circle, color=black}}


\begin{document}
	\foreach \n in {1,...,6} {
		\begin{tikzpicture}[scale=0.75, main, line join=bevel]
			\clip (-4,-1) rectangle (4,6);
			\ifthenelse{\n = 1}{}{
				\draw[main] (2.5,4.5) node [right]{\sffamily \bfseries \LARGE \pgfmathdiv{(\n-1) * (\n)}{2}\pgfmathresult};
			}
			\begin{scope}[xshift=-0.5cm]
				\foreach \i in {1, ..., \n}{
					\foreach \j in {1,...,\i} {
						\ifthenelse{\i=\n}{
							\fill[BurntOrange] (\j-\i/2,\n-\i) circle (0.45cm);
							\draw[BurntOrange] (2.75,3.5) node[right] {\sffamily \bfseries \LARGE $+$\n};
						}{
							\fill[main] (\j-\i/2,\n-\i) circle (0.45cm);
						}
										
					}
				}
			\end{scope}			
		\end{tikzpicture}
		
		\begin{tikzpicture}[scale=0.75, main, line join=bevel]
			\clip (-4,-1) rectangle (4,6);
			\draw[main] (2.5,4.5) node [right] {\sffamily \bfseries \LARGE \pgfmathdiv{\n * (\n+1)}{2}\pgfmathresult};
			\begin{scope}[xshift=-0.5cm]
				\foreach \i in {1, ..., \n}{
					\foreach \j in {1,...,\i} {
						\ifthenelse{\i=\n}{
							\fill[main] (\j-\i/2,\n-\i) circle (0.45cm);
						}{
							\fill[main] (\j-\i/2,\n-\i) circle (0.45cm);
						}
						
					}
				}
			\end{scope}			
		\end{tikzpicture}
	}
\end{document}